\chapter{Control Flow}

\section{If Statements}
An \verb!if! statement is a block of code that can be executed by a program if 
a condition results to \verb!true!. For example let's ask the user for a number 
from 1 to 10, if it's less than 5, we will say \verb!"Wow that's low huh!"!. 


\begin{lstlisting}
use std::io::stdin; //To get standard input 

fn main(){
    let mut input = String::new();
    println!("Enter a number from 1 to 10!");
    stdin().read_line(&mut input).unwrap(); //Gets user input 

    //Now to convert the String to i32
    let i:i32 = input.trim().parse().expect("Expected an integer");

    if i < 5 {
        println!("Wow that's low huh!");
    }
}    
\end{lstlisting}

\begin{remark}
    \par As you can see, we have the \verb!if! keyword followed by the condition, such as \verb!i < 5! 
    then the block of code executed is surrounded by \verb!{ }!. 
\end{remark}

\section{Else Clause}
\par This program isn't really that good, it only gives you something if you guess a number less than 5, but let's say I guess 8? I should still get something, and that's when the \verb!else! clause comes in, unlike the \verb!if! statement, it will execute a block of code 
only if it results to \verb!false!. So let's fix our program so when I guess 5 or greater I am presented with \verb!"Buddy you guessed too high!"!. 

\begin{lstlisting}
use std::io::stdin; //To get standard input 

fn main(){
    let mut input = String::new();
    println!("Enter a number from 1 to 10!");
    stdin().read_line(&mut input).unwrap(); //Gets user input 

    //Now to convert the String to i32
    let i:i32 = input.trim().parse().expect("Expected an integer");

    if i < 5 {
        println!("Wow that's low huh!");
    } else {
        println!("Buddy you guessed too high!");
    }
}    
\end{lstlisting}

\begin{remark}
Notice how when you're using the \verb!else! clause, you don't require a 
condition, since it's like a default.    
\end{remark}


\section{Else if Statement}
Now I got another problem with my program, I want something to happen when they guess 
5, so how do I do that, add another \verb!if! statement? But that would look so disgusting, 
so instead why not add an \verb!else if! statement? An \verb!else if! statement is like a second order condition, 
if the first \verb!if! statement results to false, then the \verb!else if! statement is checked, and if it results to \verb!true! then it's block of code is executed. 

\begin{lstlisting}
use std::io::stdin; //To get standard input 

fn main(){
    let mut input = String::new();
    println!("Enter a number from 1 to 10!");
    stdin().read_line(&mut input).unwrap(); //Gets user input 
    
    //Now to convert the String to i32
    let i:i32 = input.trim().parse().expect("Expected an integer");
    
    if i < 5 {
        println!("Wow that's low huh!");
    } else if i == 5{
        println!("You guessed right!!!");
    } else {
        println!("Buddy you guessed too high!");
    }
}
\end{lstlisting}

\begin{remark}
\par An \verb!else if! statement has the same syntax as an \verb!if! statement, except it must be 
after an \verb!if! statement and before the \verb!else! clause. 
\end{remark}