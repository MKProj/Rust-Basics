\chapter{Imports \& Namespaces}
\par In many languages it is common to import libraries from external, or standard libraries, in our case we will only consider the standard library until we talk about [Cargo in Section 6](06-cargo.md). So I guess the best way to start is by first figuring out how to import a library, and that is done by the \verb!use! keyword. 

\par To help with this section, we will create a simple user input program by using the standard library input/output module ,!std::io!. 

\begin{lstlisting}
use std::io;

fn main() {
    let mut input = String::new(); // Creates a new String
    io::stdin().read_line(&mut input).unwrap();
    println!("Your input is {}", input);
    // What this means is that we borrow (&) the binding input, and
    // it must be mutable, since it will change it's value
    
}    
\end{lstlisting}

\par In this example, we specify that we only want to use the \verb!io! module in the 
 \verb!std! library. However let's say we don't know what we really want from the \verb!std! 
library, and instead of guessing, we could just import everything by using \verb!*! after 
 \verb!std::!. 

\par Consider the example below: 
\begin{lstlisting}
use std::*;
fn main() {
    //We will be copying one file to another
    let mut a = fs::File::create("file.txt")
        .expect("file not created");
    //The file we will be copying from
    let mut b = fs::File::open("copy.txt")
        .expect("file not found");
    //Using io to copy b to a
    io::copy&mut b, &mut a)
        .expect("file cannot be copied");
}    
\end{lstlisting}
\par We get the following result:
\lstset{language=sh}
\begin{lstlisting}
$ rustc ex.rs
$ ./ex

#copy.txt
This file has words!!!

#file.txt
This file has words!!!    
\end{lstlisting}


As you can see we could access anything from \verb!std! using \verb!*!, however it isn't the
most effective thing to do, in this example we only needed \verb!io! and \verb!fs! modules, so 
let's just import those. 
\lstset{style=mystyle}
\begin{lstlisting}
use std::{fs, io};

fn main() {
    //We will be copying one file to another
    let mut a = fs::File::create("file.txt")
        .expect("file not created");

    //The file we will be copying from
    let mut b = fs::File::open("copy.txt")
        .expect("file not found");
    //Using io to copy b to a
    io::copy(&mut b, &mut a).expect("file cannot be copied");
} 
\end{lstlisting}

As you can see, if you know what you want to import, and need multiple modules, then
surround them with \verb!{ }!. 

\begin{remark}
\par When it comes to namespaces, whatever the module you import is the name
you must start with. Such as, \verb!std::io! you use \verb!io::!, but if you did \verb!std::fs::File!, you must start with \verb!File::!
\end{remark}
